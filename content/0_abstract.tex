\thispagestyle{empty}
\section*{Kurzdarstellung}
\label{sec:kurzdarstellung}

Die vorliegende wissenschaftliche Arbeit behandelt die Analyse und anschließende Überarbeitung des User Interfaces von EB GUIDE Studio 6, mit der Zielsetzung dessen Usability zu verbessern.
Um dies zu erreichen wurde sich an den einzelnen Iterationsschritten des Human-Centered Design Process orientiert.
Für die Identifizierung der Schwächen im Interface wurden Modellierer innerhalb der Zielgruppe bei ihrer täglichen Arbeit beobachtet und befragt.
Für drei dieser Schwächen wurden, nach allgemein gültigen Gestaltprinzipien, Verbesserungen erarbeitet, welche teilweise mithilfe eines Prototyping Tools und teilweise, im bestehenden Projekt, mit C\# und WPF umgesetzt wurden.
Die durchzuführende Testaufgabe wurde so ausgelegt, dass die Nutzer, während des Tests, mit allen eingearbeiteten Verbesserungen interagieren.
Um vergleichbare Werte zu erhalten wurde die Aufgabe von je fünf Nutzern mit dem bestehenden und dem überarbeiteten Interface durchgeführt.
Um die Usability zu messen wurde bei der Auswertung der Tests auf die Effizienz und Fehlerrate der Nutzer geachtet.

Nach einer Iteration des Design Process wird deutlich, dass die Anpassungen die Usability teilweise erhöht haben, die Verbesserungen jedoch auch noch Schwächen enthalten die Nachbesserung verlangen.
Hierfür können, aufbauend auf den durch diese Arbeit bereit stehenden Grundlagen, weitere Iterationen des Design Process durchgeführt werden, bis die Verbesserungen die Benutzeranforderungen hinreichend erfüllen.


