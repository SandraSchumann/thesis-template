\thispagestyle{empty}
\section*{Kurzdarstellung}
\label{sec:kurzdarstellung}

Die vorliegende wissenschaftliche Arbeit behandelt die Analyse und anschließende, teilweise Überarbeitung des User Interfaces von EB GUIDE Studio 6, mit der Zielsetzung dessen Usability zu verbessern.
Um dies zu erreichen, wird sich an den einzelnen Iterationsschritten des Human-Centered Design Process orientiert.
Für die Identifizierung der Schwächen im Interface werden Modellierer innerhalb der Zielgruppe bei ihrer täglichen Arbeit beobachtet und befragt.
Für drei dieser Schwächen werden, nach allgemein gültigen Gestaltprinzipien, Verbesserungen erarbeitet, welche teilweise mithilfe eines Prototyping Tools und teilweise, im bestehenden Projekt, mit C\# und WPF umgesetzt werden.
Die zu bearbeitende Testaufgabe wird so ausgelegt, dass die Nutzer, während des Tests, mit allen eingearbeiteten Verbesserungen interagieren.
Um vergleichbare Werte zu erhalten, wird die Aufgabe von je fünf Nutzern mit dem bestehenden und dem überarbeiteten Interface durchgeführt.
Bei der Messung der Usability, wird bei der Auswertung der Tests auf die Effizienz und Fehlerrate der Nutzer geachtet.

Nach einer Iteration des Design Process wird deutlich, dass die Anpassungen die Usability teilweise erhöht haben, die Verbesserungen jedoch noch Schwächen enthalten, die Nachbesserung verlangen.
Hierfür können, aufbauend auf der, durch diese Arbeit bereit gestellten Grundlage, weitere Iterationen des Design Process durchgeführt werden, bis die Verbesserungen die Benutzeranforderungen hinreichend erfüllen.


