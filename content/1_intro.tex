\chapter{Einleitung}\label{ch:intro}

In diesem einführenden Kapitel wird zunächst kurz das Partnerunternehmen der Abschlussarbeit mit der zugehörigen Abteilung User Experience vorgestellt.
Weiterhin wird die Motivation dieser Arbeit und das verfolgte Ziel erläutert, bevor es noch einen Überblick über die folgenden Kapitel gibt.

\section{Elektrobit Automotive GmbH}
Partner der Bachelorarbeit ist die Firma Elektrobit Automotive GmbH - im Folgenden nur noch als EB bezeichnet.
EB ist ein vielfach ausgezeichnetes, internationales Unternehmen, welches sich auf die Entwicklung von Produkten und Dienstleistungen im Bereich der Automobilindustrie spezialisiert hat.
Mit über 30 Jahren Branchenerfahrung bietet EB seinen Kunden unter anderem innovative Lösungen für das vernetzte Fahrzeug, Human Machine Interface Technologien (HMI), Navigations- und Fahrassistenzsysteme und Steuergeräte. 
Die Automotive Software von EB befindet sich in über 1 Billionen Geräten die in mehr als 90 Millionen Fahrzeugen weltweit Verwendung finden.
Mit über 2300 beschäftigten Mitarbeitern, verteilt auf 3 Kontinente und 9 Länder, und einer durchschnittlichen jährlichen Wachstumsrate von über 10 \% ist EB ein weltweit etabliertes Unternehmen mit Hauptsitz in Erlangen\cite{about_eb}.

\section{Abteilung User Experience}
Jedes Gerät das für den alltäglichen Gebrauch gedacht ist, sollte eine erfolgreiche Interaktion gewährleisten.
Dafür ist eine Schnittstelle zwischen Mensch und Maschine (HMI) die einen intuitiven, einfachen und schnellen Umgang mit diesem Gerät ermöglicht unabdingbar.
Um die Erfahrungsqualität im Allgemeinen möglichst hoch zu halten wünschen Nutzer sich auf ihre Bedürfnisse angepasste User-Interfaces in allen Lebensbereichen, womit der Bereich UX auch in der Automobilbranche einen hohen Bedeutungsgrad genießt.
Die Abteilung User Experience von EB befasst sich vor allem mit der Entwicklung multimodaler HMIs für Kombiinstrumente, Head Units und Head-Up Displays.
Diese werden von EB von der Konzeptphase bis hin zur Serienentwicklung mit Hilfe von EB-GUIDE entwickelt.

\section{Motivation}
Die in dieser Arbeit untersuchte Software, EB GUIDE, wird sowohl intern bei Elektrobit genutzt, als auch extern als Produkt vertrieben.
Bei firmeninterner Nutzung eines Produktes hängt die Usability des Selbigen direkt mit der Effektivität der damit durchgeführten Arbeit zusammen.
Dies begründet sich darin, dass Zeit, die Nutzer damit verbringen unklare Funktionen der Software zu verstehen, bezahlte Arbeitszeit darstellt ist der jedoch kein tatsächlicher Arbeitsfortschritt zu verzeichnen ist.
Darüber hinaus führt das fehlerhafte ausführen, oder nicht auffinden von Softwarefunktionen zu steigender Frustration des Nutzers, was ebenfalls die Produktivität negativ beeinflusst.

Für den Vertrieb einer Software ist schlechte Usability ebenfalls fatal.
Sobald es mehr als eine Software für ein Anwendungsgebiet gibt, werden Firmen die Software nutzen, die eine bessere Usability aufweist.
Diese Entscheidung begründet sich auf den gleichen Argumenten, weshalb Elektrobit bei  firmeninterner Nutzung eine Software mit hoher Usability bevorzugt.

Eine schlechte Usability von EB Guide hat also sowohl interne, als auch vertriebliche negative Folgen, weshalb Elektrobit es anstrebt die Usabiity der Software durchgehend zu verbessern.

\section{Zielsetzung}
Ziel dieser Bachelorarbeit ist es, Schwachstellen im User Interface von EB GUIDE zu identifizieren und durch deren Verbesserung die Usability von EB GUIDE zu erhöhen.
Dafür ist zuerst eine Analyse der Arbeitsabläufe innerhalb der Modellierungsarbeit nötig um entsprechende Probleme im User Interface zu erkennen.
Anschließend gilt es Konzepte zu entwickeln welche, durch Anpassung und Überarbeitung der entsprechenden Komponenten in der Benutzerschnittstelle, diese Probleme beheben oder minimieren.
Diese Konzepte werden anschließend noch grafisch und interaktiv visualisiert, um abschließende Usability-Tests durchführen zu können.
Dabei wird eine identische Aufgabenstellung von Probanden mit dem Alten und dem Überarbeiteten Interface durchgeführt und durch den Vergleich der Ergebnisse festgestellt, ob die Anpassungen die Usability von EB Guide erhöht haben.

\section{Aufbau der Arbeit}
Auf den folgenden Seiten werden zunächst die Theoretischen Grundlagen erläutert, die benötigt werden um eine Usabilitystudie durchzuführen und zu verstehen.
Dazu zählt auch das Prinzip des Human-Centered Design Process, nach dessen Iterationsschritten die Kapitel dieser Arbeit grob gegliedert sind.
Anschließend an die theoretischen Grundlagen werden Analysen an der bestehenden Software und am Arbeiterverhalten des Nutzer durchgeführt, um damit die existierenden Schwächen des Interfaces festzulegen und Verbesserungen erarbeiten zu können.
Diese werden im darauffolgenden Kapitel, in Form eines Prototyps umgesetzte oder direkt implementiert um die abschließenden Usability Tests durchführen zu können.
Abschließend werden die Ergebnisse dieser Tests ausgewertete und es wird ein kurzer Ausblick gegeben, welche Punkte in folgenden Iterationen des Design Process weiter verfolgt werden können.
