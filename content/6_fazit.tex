\chapter{Fazit}\label{ch:summary}

Human-Centered Design Process wurde einmal durchlaufen
eine Steigerung der Usability konnte vrzeichnet werden, auch wenn nicht alle definierten Benutzeranforderungen zufriedenstellend erfüllt wurden
ist für den Umfang einer Iteration ein zu erwartendes Ergebnis

Um Usability messbar zu machen wurden Benutzeranforderungen definiert
Dies geschah auf der Grundlage von Untersuchungen
	Beobachtung der Zielgruppe bei ihrer täglichen Arbeit, hierbei Definition von
		der Arbeitsumgebung (EB GUIDE)
		Nutzergruppe (Mit der die Tests anschließend durchgeführt werden)
		gewohnte Arbeitsaufgaben der Nutzer

Für beobachtete Schwächen werden qualitative Nutzeranforderungen definiert
Von allen 5 Schwächen wurden drei ausgewählt 
	nach Kriterien:
		Mischung aus nach einer Iteration bereits Ergebnisse und weitere Verfolgung der Arbeit für Elektrobit möglich
			von daher zwei kleinerer und eine große Anpassung
		Zwei Prinzipien der Usability werden untersucht
			Effizienz und Fehlerrate
		Anpassungen die den Nutzern aktuell den größten Mehrwert liefern

	Auswahl:
		Template Properties puplish to template interface eignet sich
			Zeit gut messbar, Fehlerrate ebenfallse und Nutzer können zusätzlich gut befragt werden welche Position sie vorziehen
			kleine Anpassung
	
		Filter der Widget Feature Properties
			Mehrwert absehbar, Zeit und Fehlerrate gut messbar
			großer Mehrwert für die Nutzer
			kleine Anpassung

		Mehrfachselektion	
			noch keine Unterstützung dieser Funktionen für EB GUIDE
			Funktionalität ist man jedoch aus anderer Software gewohnt
			Fehlerrate und Zeit gut messbar
			große Anpassung

Beim Design dieser drei Verbesserungen wurde sich an bereits bestehenden Elementen in EB GUIDE oder in anderer Fremdsoftware gehalten

Für die Umsetzung der Entwürfe mussten Interaktionsmöglichkeiten definiert werden die vom Prototyp oder der Software bereitsgestellt werden müssen
	
	Bei Prototyp vor allem wichtig grundlegedne Funktionen von EB GUIDE zur Verfügung zu stellen

	Puplish to template interface muss an der neuen Position funktionieren
	
	Filter muss die Widget Feature Properties live nach der Eingabe des Nutzers filtern

	Mehrfachselektion
		Panel für gleichzeitig ausgewähle Elemente, gemeinsam im View bewegbar sein, Alignment Actions und Inster in Template

	Prototype mithilfe von AXURE RP umgesetzte

	FIlter mithilfe von WPF und C\# umgesetzt, von bestehender Projektstruktur vorgegeben, mithilfe von MVVM Pattern gelöst

Test mit Lookback durchgeführt
	
	Remote Usability Test

	Als Aufgabe die Modellierung einer Starstscreens wie aus HMI im Auto gewohnt
		Hierfür noch quantitative Benutzeranforderungen formuliert mit denen die Usability gemessen werden kann

	Arbeitsaufgabe als Styleguide und textuelle Angabe
		Test mit 10 Nutzern und zwei verschieden gestellten Aufgaben um vergleichen zu können

Ergebnisse
	Filter steigert Effizienz, führt jedoch zu Fehlern

	Publish to template interface, verringerte Fehlerrate und gesteigerte Effizienz

	Multiselektion
		Insert in Template verworfen
		Alignment Actions steigern Effzienz, erhöhen jedoch auch Fehlerrate
		Gemeinsame Positionierung schneller, und Fehlerrate von 0
		Skalierung aktuell kein Mehrwert über Multiselektion


Ergebnisse können in weiteren Iterationen verfolgt und ausgearbeitet werden
		
	

	
			
		
