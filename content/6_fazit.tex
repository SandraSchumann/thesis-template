\chapter{Fazit}\label{ch:summary}

Im Rahmen dieser Bachelorarbeit, wurde eine Iteration des Human-Centered Design Process durchlaufen.
Hierbei wurden Usabilityschwächen im Interface von EB GUIDE 6 entdeckt, woraufhin für drei dieser Defizite Lösungsansätze erarbeitet wurden.
Durch diese eingearbeiteten Verbesserungen, konnte eine Steigerung der Usability verzeichnet werden, auch wenn nicht alle definierten Benutzeranforderungen zufriedenstellend erfüllt werden konnten.
Für den zeitlichen Rahmen, in dem nur eine Iteration des Design Process vorgesehen war, entspricht dieses Ergebnis jedoch den Erwartungen, und kann mithilfe weiterer Iterationen noch verbessert werden.

Die, für die Bewertung der abschließenden Messungen benötigten Benutzeranforderungen, wurden hierbei auf der Grundlage von durchgeführten Untersuchungen der Arbeitsabläufe von Nutzern formuliert.
Während dieser Untersuchungen, wurden Personen aus der zutreffenden Zielgruppe, bei ihrer täglichen Arbeit beobachtet.
Dadurch konnten repräsentative Schwachstellen im User Interface von EB GUIDE aufgedeckt werden, für die qualitative Nutzeranforderungen definiert werden konnten.
Aus allen gefunden Schwächen wurden drei ausgewählt, wobei darauf geachtet wurde eine Anpassung mit großem, und zwei Anpassungen mit kleinem Umfang zu wählen, anhand deren Usabilitymerkmale, Effizienz und Fehlerrate überprüft werden können.

Zu diesem Zeitpunkt der Arbeit war noch nicht absehbar, dass ein Teil der ausgewählten Anpassungen mithilfe eines Prototyps und ein Teil in EB GUIDE implementiert umgesetzt werden würde.
Diese Tatsache führte bei dem abschließenden Test zur Notwendigkeit diesen in zwei Teile aufzuspalten, was für die Nutzer, durch die Unterbrechung des Arbeitsablaufes.
Daher wäre es bereits an dieser Stelle hilfreich gewesen, sich über die Umsetzung der Anpassungen Gedanken zu machen, und nur solche zu wählen, die einen zusammenhängenden Testablauf ermöglichen.

Bei der Überarbeitung der ausgewählten Schwächen - der Funktion  \glqq puplish to template interface\grqq{}, den Filter für die \glqq Widget Feature Properties\grqq{} und die \glqq Mehrfachselektion\grqq{} - wurde sich beim Entwurf des Designs, an bekannten Gestaltprinzipien orientiert.
Für die Umsetzung dieser Entwürfe mussten zusätzlich Interaktionsmöglichkeiten definiert werden, die der Prototyp bzw. Software bereitstellen muss, um eine erfolgreiche Interaktion zu gewährleisten.
Im Rahmen der durchgeführten Tests wurde deutlich, dass zwar nicht alle gewünschten Interaktionen durchgeführt werden konnten, es jedoch für jeden Nutzer möglich war das Ziel der Arbeitsaufgabe zu erreichen.
Allgemein ist es - bei einer so umfangreichen Software wie EB GUIDE - nicht möglich oder nötig alle Funktionen derselbigen innerhalb eines Prototyps zu simulieren.
Da alle Nutzer ihr verfolgtes Ziel erreichen konnten, lässt sich jedoch festhalten, dass die Interaktionsmöglichkeiten die Bedürfnisse der Nutzer ausreichend abgedeckt haben, ohne zu großen Aufwand für den Modellierer des Prototyps zu erzeugen.

Die Umsetzung - in Form eines Prototyps - wurde mit AXURE RP gelöst, was eine Einarbeitung in dieses Tool erfordert hat.
Da die Dokumentation hierzu jedoch sehr umfangreich ist und auch Hilfestellung von Mitarbeitern von Elektrobit möglich war, war die Software, rückwirkend betrachtet, gut geeignet.
Auch war es möglich alle definierten Interaktionsmöglichkeiten - aufgrund der Möglichkeit Logik und globale Variablen in den Prototyp einzubauen - wie gewünscht umzusetzen.
Die Implementierung des Filters wurde mithilfe von WPF und C\# gelöst, wobei bereits im Studium erlangte Kenntnisse zu diesen Programmiersprachen genutzt werden konnten.
Deshalb war es lediglich nötig, sich in das Konzept des MVVM Pattern einzuarbeiten, sowie die Struktur des bestehenden Projektes zu verstehen.

Die abschließenden Tests wurden in Form von Remote Usability Tests umgesetzt.
Aufgrund der örtlichen Verteilung der Probanden war dies eine offensichtliche Wahl, die auch zu gut vergleichbaren Ergebnissen bei der Auswertung führte.
Die Wahl der Arbeitsaufgabe kann insofern positiv bewertet werden, dass die Nutzer intuitiv wussten was sie tun sollten und in welchem Zusammenhang die textuelle Angabe und der Styleguide zu betrachten sind.

Als Ergebnisse der Tests kann abschließend festgehalten werden, dass der Filter der \glqq Widget Feature Properties\grqq{} zwar die Effizienz steigert, jedoch aktuell noch zu Fehlern führt.
Die neue Position der Funktion \glqq Publish to template interface\grqq{}, verringert die Fehlerrate der Nutzer und steigert gleichzeitig die Effizienz, weshalb diese Anpassung als abgeschlossen betrachtet und implementiert werden kann.
Im Rahmen der Multiselektion kann die Funktion \glqq Insert in Template\grqq{} verworfen werden, da die Probanden diese Funktion einstimmig als nicht intuitiv eingestuft haben.
Die \glqq Alignment Actions\grqq{} steigern aktuell die Effizienz, erhöhen jedoch - vermutlich aufgrund der Fremdheit der neuen Funktionalität - aktuell noch die Fehlerrate.
Die gemeinsame Anpassung von Eigenschaften ausgewählter Elemente, bringt bei der Positionierung einen Mehrwert der Effizienz und senkt die Fehlerrate, bei der Skalierung von Objekten ist aktuell jedoch keine Verbesserung erkennbar.

Aufgrund der Tatsache, dass der Prototyp einsatzfähig vorliegt, können damit noch weitere Iterationen des Design Process durchlaufen werden und weiter auf die, noch bestehenden Usabilityschwächen eingegangen werden.
Es wurden mit dieser Arbeit also zum einen Ergebnisse erlangt, die in EB Guide als Verbesserung implementiert werden können.
Zum anderen wurde eine Grundlage geschaffen um weitere Iterationen durchlaufen zu können und die große Anpassung der Multiselektion noch um weitere Inhalte ergänzen und testen zu können.
